\label{Algorithm}
\section{Algorithm}

The following section covers how the algorithm works.
Let us start by defining some notations.

\subsection{Notations}
Let G = (V,E) be an weighted, undirected simple graph and let n = \(\lvert V\rvert\) and m = \(\lvert E\rvert\).

A vertex v denotes an actor. Any edge e between vertices \(v_1\) and \(v_2\) denotes a set of movies these two actors have played together. Weight of the edge, W(e) denotes the size of that set.

Denote by A(v) the set of adjacent edges to vertex v.

\(\Pi\)(e) is the set of (two) vertices adjacent to an edge e.

SET(\(v_1, v_2,... v_n)\) - returns a set of unique elements.

MovieCount denotes the biggest number found so far of common movies between any given three actors.
\subsection{Pseudocode}



\begin{verbatim}
1	moviesCount ← 0;
2	{a_1,a_2,a_3\)};
3	    FOR v ∈ V DO
4	        FOR i ← 0 to size of A(v) DO
5		    FOR j ← i + 1 to size of A(v) DO
6	                e1←A(v)[i];
7	                e2←A(v)[j];
8	                IF(MoviesCount < MIN(W(e1), W(e2))) THEN
9	                count ← CommonMovieSubsetCount(e1, e2);
10	                IF(moviesCount < count) THEN
11	                movieCount ← count;
12	              
13	                
	            
	            
\end{verbatim}

