\label{Problem Description}
\section{Problem Description}

The \textbf{goal} is to find the three actors, whose movie count they have played together in is maximized among he whole dataset.

\textbf{Input} is a list of actor-movie pair, for each actor that has played in a movie.

\textbf{Output} is a list of actors with the desired property, on an empty list if no three actors have played together in the same movie.
\\
\\
The algorithm we are presenting works on two main steps:
\begin{itemize}
  \item Build the data structure - the efficiency of the algorithm is determined by the data structure it runs on. On the other hand, the data structure is specifically designed to solve this problem. Try to solve another problem, say maximizing the number of movies 4 actors have played together in, would require a different  structure and approach altogether.
  \item Traverse data structure and output result - the algorithm works by  discarding part of the information on every iteration, thereby reducing the size of the problem.
\end{itemize} 

More details on the algorithm inner workings in the next section.




ALGORITHM SECTION
As mentioned in the previous section, the algorithm works starts by frist building the data structure. The data structure is a graph, where vertices represent actors and edges between them the movie(s) these actors played together in. Figure XXX represents the data structure.
//
//
//
To clearly illustrate the problem, the picture above represents a multi-graph, i.e. there can be multiple edges between two vertices. This is not the case of the actual data structure, however, as multiple edges are collapsed into a single one, where the weight is the sum of the weights of the original edges. The weight of an edge is the initially 1 - a single movie common to two actors (vertices). 

After the structure is constructed, the actual data processing takes place.

Iterate over all actors
If there are more than 2 edges,
get the min