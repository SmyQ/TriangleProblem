\documentclass[11pt]{article} % use larger type; default would be 10pt

\usepackage{mathtools}
\usepackage{amsmath}
\usepackage[noend]{algpseudocode}
\usepackage{graphicx}

\usepackage[utf8x]{inputenc} % set input encoding (not needed with XeLaTeX)

%%% Examples of Article customizations
% These packages are optional, depending whether you want the features they provide.
% See the LaTeX Companion or other references for full information.

%%% PAGE DIMENSIONS
\usepackage{geometry} % to change the page dimensions
\geometry{a4paper}

\usepackage{graphicx} % support the \includegraphics command and options

% \usepackage[parfill]{parskip} % Activate to begin paragraphs with an empty line rather than an indent

%%% PACKAGES
\usepackage{booktabs} % for much better looking tables
\usepackage{array} % for better arrays (eg matrices) in maths
\usepackage{paralist} % very flexible & customisable lists (eg. enumerate/itemize, etc.)
\usepackage{verbatim} % adds environment for commenting out blocks of text & for better verbatim
\usepackage{subfig} % make it possible to include more than one captioned figure/table in a single float
% These packages are all incorporated in the memoir class to one degree or another...

%%% HEADERS & FOOTERS
\usepackage{fancyhdr} % This should be set AFTER setting up the page geometry
\pagestyle{fancy} % options: empty , plain , fancy
\renewcommand{\headrulewidth}{0pt} % customise the layout...
\lhead{}\chead{}\rhead{}
\lfoot{}\cfoot{\thepage}\rfoot{}

%%% SECTION TITLE APPEARANCE
\usepackage{sectsty}
\allsectionsfont{\sffamily\mdseries\upshape} % (See the fntguide.pdf for font help)
% (This matches ConTeXt defaults)

%%% ToC (table of contents) APPEARANCE
\usepackage[nottoc,notlof,notlot]{tocbibind} % Put the bibliography in the ToC
\usepackage[titles,subfigure]{tocloft} % Alter the style of the Table of Contents
\renewcommand{\cftsecfont}{\rmfamily\mdseries\upshape}
\renewcommand{\cftsecpagefont}{\rmfamily\mdseries\upshape} % No bold!
%%% END Article customizations

%%% The "real" document content comes below...
\title{Three actors playing together in most movies}
\author{Plamen Kmetski (pkme)\\Kamil Androsiuk (kami)\\Agnieszka Majkowska (amaj)}
%\date{} % Activate to display a given date or no date (if empty),
         % otherwise the current date is printed

\begin{document}
\maketitle

\begin{abstract}

Large data is difficult to process. The main difficulty comes from the fact it is not possible to store the data in its entirety in memory and directly manipulate it. A number of algorithms, each of them with its own strategy, are designed for working around this problem. One such strategy is splitting the data into bits that can be computed independently. Alternatively, another strategy offers an approximation of the result with the advantage of using little memory. In this paper we explore two such algorithms, comparing their running times and space usage. The first algorithm, designed by us, is specifically targeted for solving the problem at hand. This makes it possible to apply some seemingly small tweaks, which increase performance dramatically. The second is the Misra-Gries streaming algorithm. Furthermore, the concept of parallelism is explored in the context of our solution. It is shown that with very little additional code the sequential algorithm can be easily parallelised.

\end{abstract}

\section*{Keywords}
\label{Keywords}
IMDB, graph, triplets count, streaming, Misra-Gries, parallelism


\label{Introduction}
\section{Introduction}

The IMDB dataset provides data about actors, movies and relations between them. Based on this dataset a number of interesting properties can be extracted from the data. The point, however, is doing this in efficient manner.
\label{Problem Description}
\section{Problem Description}

The \textbf{goal} is to find the three actors, whose movie count they have played together in is maximized among he whole dataset.

\textbf{Input} is a list of actors, movies and actor-movie pair, for each actor that has played in a movie.

\textbf{Output} is a list of actors with the desired property, on an empty list if no three actors have played together in the same movie.
\\
\\
The algorithm we are presenting works on two main steps:
\begin{itemize}
  \item Build the data structure - the efficiency of the algorithm is determined by the data structure it runs on. On the other hand, the data structure is specifically designed to solve this problem. Try to solve another problem, say maximizing the number of movies 4 actors have played together in. The same data structure can be used for this problem, but a different algorithm altogether.
  \item Traverse data structure and output result - the algorithm works by  discarding part of the information on every iteration, thereby reducing the size of the problem.
\end{itemize} 

More details on the algorithm inner workings in the next section.




ALGORITHM SECTION
As mentioned in the previous section, the algorithm works starts by frist building the data structure. The data structure is a graph, where vertices represent actors and edges between them the movie(s) these actors played together in. Figure XXX represents the data structure.\vspace*{3\baselineskip}
To clearly illustrate the problem, the picture above represents a multi-graph, i.e. there can be multiple edges between two vertices. This is not the case of the actual data structure, however, as multiple edges are collapsed into a single one, where the weight is the sum of the weights of the original edges. The weight of an edge is the initially 1 - a single movie common to two actors (vertices). 

After the structure is constructed, the actual data processing takes place.

Iterate over all actors
If there are more than 2 edges, get the edge with minimal weight - the one having a collection of least movies. As we are ultimately looking for a common set of two subset (the movies from two edges), we can use the one with the least movies.
Having the minimal weight edge, try finding the common set with the rest of the edges. After all the edges are traversed, get the common set of maximum size, this will be the set which the three vertices have in common. At the end, remove any references to these edges from these three vertices. This will make future iterations faster, as less edges need to be examined.

Advance in the vertices list and repeat the above operation again with the new vertex.
\section{Standard approach}
\label{Standard}

We started with implementing a data streaming technique. Since our goal is to find heavy hitters, we chose Misra-Gries algorithm to save space and to process large amount of data fast.

\subsection{Pseudocode}
\label{MisraGries}
The idea of our version of Misra-Gries is to process data movie by movie. Each movie has a list of actors and while processing them we generate all triplets for each movie. Then we maintain a hash table which contains pairs of triplets and their respective count indicating relative number of occurences of the triplet. We add a triplet to the table if it does not appear there yet or we increment count of analyzed triplet. The result of the analysis is the triplet with the highest count. For more details we present pseudocode of the algorithm in paragraph below.
\\
Let m denote number of movies and A(i) denotes list of actors playing in a given movie i.
Let H be a hash table containing k pairs: triplet (t) and its count (\(count_t)\).
\begin{verbatim}
Algorithm 1: MisraGries()
1	FOREACH movie DO
2	  FOR i ← 0 to size of A(movie) DO
3	    FOR j ← i + 1 to size of A(movie) DO
4	      FOR k ← j + 1 to size of A(movie) DO
5	        IF {A(movie)[i],A(movie)[j],A(movie)[k]} ∈ H THEN
6	          count[t] += 1;
7	        ELSE
8	          INSERT({A(movie)[i],A(movie)[j],A(movie)[k]}, count);  
9	        END IF
10	       IF k < H.length THEN
11	       FOREACH t IN H DO
12	         count[t] -= 1;
13	          IF count[t] = 0 THEN
14	            H.REMOVE(t);
15	          END IF
16	        END FOREACH
17	      END IF      
18	    END FOR
19	  END FOR
20	 END FOR
21	END FOREACH
22	RETURN H.MAX(count);	  	                    	  
\end{verbatim}

The algorithm allows to save memory since not store all processed data is stored. The efficiency and its running time highly depend on the cache size. The bigger it is, the slower algorithm computes and the more memory space we use. On the other hand with bigger table, we can work with bigger sample, which would result in an approximation that is closer to the actual answer.
\\
The cache size was determined based on calculations and experiments. We estimated that with IMDB database we have nearly 2 billion triplets for all the movies. We ran multiple expreiments with different cache sizes, and the number we found most acceptable in terms of running times was 20 000 triplets. This represents 0.001\% of the whole set. While the sample is relatively small, the running time with any bigger cache size would have been unacceptable. More on the running times in section \ref{Experiments}
\\
\subsection{Analysis}
\label{AnalysisMisraGries}
Memory usage in Misra-Gries algorithm is very little, the space is needed only for the hash table. Complexity of the algorithm highly depends on the size of cache and the dataset to be processed. The bigger they are, the longer computation time is needed. The complexity of the algorithm is \(\sum\limits_{i=1}^m{a_i \choose 3}*k\) where k is size cache size. We can reduce it to \(\sum\limits_{i=1}^m{a_i^3}*k\).
\label{Algorithm}
\section{Algorithm}

The algorithm we are presenting works on two main steps:
\begin{itemize}
  \item Build the data structure - the efficiency of the algorithm is determined by the data structure it runs on. On the other hand, the data structure is specifically designed to solve this problem.
  \item Traverse data structure and output result - the algorithm works by  discarding part of the information on every iteration, thereby reducing the size of the problem.
\end{itemize} 

\subsection{Notations}
Let G = (V,E) be an weighted, undirected simple graph and let n = \(\lvert V\rvert\) and m = \(\lvert E\rvert\).

A vertex v denotes an actor. Any edge e between vertices \(v_1\) and \(v_2\) denotes a set of movies these two actors have played together. Weight of the edge, W(e) denotes the size of that set.

Denote by A(v) the set of adjacent edges to vertex v.

SET(e) is the set of (two) vertices adjacent to an edge e.

Unique(\(v_1, v_2,... v_n)\) - returns a set of unique elements.

Remove(v) - removes edges adjacent to v from the graph G.

MovieCount denotes the biggest number found so far of common movies between any given three actors.

\subsection{Data structure}
As mentioned in the previous section, the algorithm works starts by first building the data structure. The data structure is a graph, where vertices represent actors and edges between them represent the movie(s) these actors played together in. 

\begin{figure}[ht!]
\centering
\includegraphics[width=130mm]{resources/project_problem_illustration.png}
\caption{Graph Example}
\label{example}
\end{figure}

Figure \ref{example} represents the data structure. To clearly illustrate the problem, the picture above represents a multi-graph, i.e. there can be multiple edges between two vertices. This is not the case of the actual data structure, however, as multiple edges are collapsed into a single one, where the weight is the sum of the weights of the original edges. The weight of an edge is the initially 1 - a single movie common to two actors (vertices). The edge contains sorted list of the movies common for two actors adjacent to the specific edge.
\\
After the structure is constructed, the actual data processing takes place.


\subsection{Pseudocode}
Main algorithm we implemented is FindThreePlayers. After constructing the graph, we iterate over all vertices (actors). We find for each vertex subset of size of 2 of the set of adjacent edges to that vertex. We examine each pair of edges and we find the number of common movies that actors played in together (this is a common set of the subsets - the movies from two edges).  We do that only if minimal weight of edges is higher than found (by now) maximum number of movies that actors played together. If the solution is better that the already found, we save it. We continue searching and at the end, we remove any references to the adjacent edges to the analysed vertex. This will make future iterations faster, since less edges need to be examined. We go to the next iteration. Pseudocode for the algorithm is presented below:

\begin{verbatim}
Algorithm 1: FindThreePlayers()
1	moviesCount ← 0;
2	{a1,a2,a3};
3	FOR v ∈ V DO
4	  FOR i ← 0 to size of A(v) DO
5	      FOR j ← i + 1 to size of A(v) DO
6	          e1←A(v)[i];
7	          e2←A(v)[j];
8	          IF MoviesCount < MIN(W(e1), W(e2)) THEN
9	              count ← CommonMovieSubsetCount(e1, e2);
10	             IF moviesCount < count THEN
11	                  movieCount ← count;
12	                  {a1,a2,a3} ← Unique(SET(e1), SET(e2));
13	             END IF
14	          END IF
15	      END FOR
16	  END FOR
17	Remove A(v);
18	END FOR	  	                    	  
\end{verbatim}

We designed CommonMovieSubsetCount that returns number of items that are common in 2 subset given as arguments. We take advantage of the fact that the lists are sorted and we iterate over all the items in both list in linear time. The pseudocode is present below:

\begin{verbatim}
Algorithm 2: CommonMovieSubsetCount(movies1, movies2)
1	count ← 0;
2	p1 ← 0;
3	p2← 0;
4	WHILE p1 < W(movies1) AND p2 < W(movies2)
5	  IF movies1[p1] = movies2[p2] THEN
6	    INCREMENT(count);
7	    INCREMENT(p1);
8	    INCREMENT(p2);
9	  ELSE IF movies1[p1] < movies2[p2]
10	   INCREMENT(p1);
11	 ELSE IF movies1[p1] > movies2[p2]
12	  INCREMENT(p2);
13	END IF
14	RETURN count;
\end{verbatim}

\subsection{Analysis}


\label{Experiments}
\section{Experiments}


Some experiments were conducted to verify the correctness of the algorithm and its running time. The results were computed for both algorithm: "Misra-Grise"  and "Our approach".\\

Table columns description:
 
\begin{itemize}
  \item IMDB IS (IMDB Input Size) - size of subset means that this is segment number 0 out of 2
  \item RN (Roles number) - total number of roles, each role desribes two actors that played together in the same movie
  \item BDST (Building Data Structure Time) - time required to build data structure (graph)
  \item ART (Algorithm Running Time) - running time of tested algorithm
  \item RT (Result Triplet) - three actors that played tgother in bigest amount of movies
  \item RM (Result Movies) - Number of movies that triplet have been ivovled in
\end{itemize}

Resuts for Misra-Gris algrithm:

\begin{table}[h]
\begin{tabular}{|p{2.5cm}|l|p{1.9cm}|p{1.5cm}|p{4.3cm}|l|}
\hline
IMDB IS          & RN & BDST  & ART & RT                    & RM  \\ \hline
100\%            &       48 967 421      & 0 h         & 38.6 h       & \{760909, 406612, 80307\}  & 15  \\ \hline
\end{tabular}
\end{table}

Results for our algorithm:
\begin{table}[h]
\begin{tabular}{|l|l|l|l|l|l|}
\hline
IMDB IS          & RN & BDST  & ART & RT                     & RM  \\ \hline
100\%            & 48 967 421 & 393.409 sec & 8.282 sec    & \{150878, 215408, 215564\} & 130 \\ \hline
50\% (seg-0\_2)  & 24 434 967 & 200.750 sec & 4.795 sec    & \{150878, 215408, 215564\} & 130 \\ \hline
50\% (seg-1\_2)  & 24 532 454 & 196.157 sec & 3.356 sec    & \{157955, 651761, 329494\} & 101 \\ \hline
25\% (seg-0\_4)  & 12 339 915 & 98.711 sec  & 2.142 sec    & \{33500, 316918, 761020\}  & 84  \\ \hline
25\% (seg-1\_4)  & 12 273 033 & 98.430 sec  & 2.095 sec    & \{41669, 341023, 338852\}  & 94  \\ \hline
25\% (seg-2\_4)  & 12 101 052 & 95.708 sec  & 2.150 sec    & \{150878, 215408, 215564\} & 130 \\ \hline
25\% (seg-3\_4)  & 12 259 421 & 97.868 sec  & 2.061 sec    & \{157955, 651761, 329494\} & 101 \\ \hline
10\% (seg-0\_10) & 4 899 811  & 39.729 sec  & 1.225 sec    & \{33500, 316918, 761020\}  & 84  \\ \hline
1\% (seg-0\_100) & 489 009    & 4.242 sec   & 0.183 sec    & \{33500, 316918, 761020\}  & 84  \\ \hline
\end{tabular}
\end{table}

All the test have been run on the same machine. Environment specification:

\begin{description}
  \item[Processor] Intel Core i7-2760QM 2.20 GHz
  \item[Memory] 8GB, 1,333MHz DDR3
  \item[Hard drive] 1TB, 5,400rpm
\end{description}
\section{Conclusion}
\label{Conclusion}

While a streaming algorithm as Misra-Gries provides the advantage of less memory usage, it has one key disadvantage for our problem - the need to explore every possible triplet combination. In cases where the actors count of a movie is too big (1000+), this introduces an unacceptable running time. Our approach goes away with this by managing to explore a triplet only once. Furthermore, it allows for parallelisation of the computation by allowing subsets of the data to be computed independently.


\subsection{Future Work}
Possible future work might be exploring counting triangles algorithm. We can build graph in the same way we presented, but we analyze each triangle in the graph instead (a triangle represents a triplet of actors). Counting triangles method using MapReduce gives good results and might be an interesting alternative to our approach \footnote{http://theory.stanford.edu/~sergei/papers/www11-triangles.pdf}.
\input{./References.tex}

\end{document}
